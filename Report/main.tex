\documentclass[pdftex,11pt,a4paper]{article}
\usepackage[utf8]{inputenc}


\usepackage[hidelinks]{hyperref}
\usepackage[text={16cm,24cm},centering]{geometry}

\usepackage{paracol}
\usepackage{blindtext}
\usepackage{amsmath}
\usepackage[title]{appendix}

\setlength{\parskip}{1em}
%\usepackage[onehalfspacing]{setspace}

\renewcommand\thesection{\arabic{section}}
\renewcommand\thesubsection{\thesection.\arabic{subsection}}
\renewcommand\thesubsubsection{\thesubsection.\arabic{subsubsection}}

%\usepackage{natbib}
\usepackage{graphicx}
\usepackage{imakeidx}
\usepackage{changepage}   % for the adjustwidth environment
\makeindex[columns=3, title=Alphabetical Index, intoc]

\begin{document}
%\maketitle
%% Document settings
%{

\newcommand{\setlocalcolumnwidth}[1]{%
	%\let\oldratio\pcol@columnratioleft
	\setcolumnwidth{#1}%
}

\newcommand{\HRule}{\centering{\rule{1\linewidth}{1.5pt}}} % New command to make the lines in the title page

\newcommand{\decoRule}{\rule{.8\textwidth}{.4pt}} % New command for a rule to be used under figures

%}

\begin{titlepage}

\vspace{-20cm}
   \begin{center}

%      \setlength{\leftskip}{5mm}
\setlocalcolumnwidth{0.2\textwidth,0.6\textwidth,0.2\textwidth}
\begin{paracol}{3}
	
	\hspace{-1cm}\includegraphics[scale = 0.2]{eie.jpg}
	\switchcolumn
	{\LARGE School of Electrical and Information Engineering}
	         {\large \\University of the Witwatersrand} 
	         {\small \\Johannesburg, South Africa}
	         \HRule \\[0.1cm]
	         \vspace{3cm}
	\switchcolumn
		\includegraphics[scale = 0.23]{wits-orig.png}
\end{paracol}

\textbf{\Large DESIGN OF A ROBUST CODEC \\ \vspace{0.2cm} FOR A FADING CHANNEL}

\vspace{0.5cm}
         ELEN4011: DESIGN II
 
       \vspace{1.5cm}
 
       \textbf{Author:} Boikanyo Radiokana (1386807) \\
       \vspace{1.5cm}
       \textbf{Supervisor:} Prof. Fambarai Takawira
       \vspace{1cm}
       
       25 October 2019
       \vfill

\begin{adjustwidth}{5mm}{}
\textbf{Abstract:} To separate these two sections out we'll use the command which will automatically add in the amount of vertical space needed for the content to fill the page. Next we'll add in a line of text to specify what degree the thesis is being submitted for. The double backslash is used to create a new line. We'll then add more space before adding in the university logo specifying it's width as a fraction of the text width. Finally we'll add in some information about the university and the date.
\end{adjustwidth}

%\setlength{\rightskip}{-1cm}
       

       
 
   \end{center}
\end{titlepage}

\tableofcontents

\newpage
\section{Introduction}
Over the past years, the Telecommunications industry has seen a significant growth in demand for highly reliable wireless communication links that are capable of transmitting data at high rates. Unfortunately, like any other type of channel, a wireless channel has its own shortcomings. It is prone to noise, fading, reflection and shadowing \cite{22}. To combat these challenges, commonly used Forward Error Correction (FEC) techniques have been adopted in many applications to improve the reliability of data transmission. FEC enables data to be transmitted over long distances at low transmit power and at low Bit Error Rates, however, it requires an increase in channel bandwidth \cite{22}, \cite{18}.

To compensate for this, a widely used technology, MIMO architecture has been extensively explored and implemented \cite{18_1}, \cite{42}. It makes use of multiple transmit and receive antennas to improve the quality of digital communication systems. The key benefits of MIMO technology include an increase in channel capacity, high spectral efficiency as well as high data rates \cite{3}. MIMO systems have been found to have a linear increase in channel capacity with the minimum number of transmit and receive antennas \cite{42}.

In this paper, the design, analysis and simulation of a robust codec for a fading channel with MIMO architecture is presented. Section 2, 3 and 4 respectively present the design specifications, system model and the methodology adopted. Section 5, 6 and 7 details the simulation results, analysis and impacts of this type of technology on the society, the economy and the environment.




%%%%%%%%%%%%%%%%%%%%%%%%%%%%%%%%%%%%%%%%%%%%%%%%%%%%%%%%%%%%%%%%%%%%%%%%%%%%%%%
%
\section{Design Specifications}
The main objective of this project is to design a robust codec for a fading channel with an appropriate MIMO architecture that will combat the effects of fading. The following requirements and assumptions are taken into consideration when designing the codec. 

\subsection{Requirements and Success Criteria}
For the design to be deemed successful, the following requirements must be fullfilled:

\begin{itemize}
    \item The codec must achieve a spectral efficiency of at least $2bits/Hz$.
    \item The codec must be modelled for a Rayleigh fading channel.
    \item The data transmission of at least $10Mbits/sec$ must be achieved. 
\end{itemize}{}

\subsection{Assumptions}
The design is carried out with the following assumptions for simplification purposes:

\begin{itemize}
    \item Source encoding has already been implemented, therefore, the input data stream is already converted to binary format.
    \item Data encryption is not a primary component in this project, hence, it is optional.
    \item Channel State Information (CSI) is known at both the transmitter and receiver, therefore, channel estimation is not necessary.
    \item It is assumed that the relative distance between the transmitter and the receiver will mostly remain constant, thus the effects of Doppler shifts are negligible. 
\end{itemize}{}

\section{System Model}
A digital communication system  is mainly composed of three subsystems, namely: the transmitter, channel and receiver. Figure \ref{fig:1} below depicts the proposed codec design modelled from the aforementioned subsystems. In this section, a brief description of the system model is provided. The design is simulated in MATLAB R2018b.


\begin{figure}[h!]
    \centering
    \includegraphics[width=\textwidth]{system5.png}
    \caption{System Model}
    \label{fig:1}
\end{figure}{}

The source of information can be a text, video or picture and this will require binary conversion before any signal processing can take place. This process of conversion includes reducing the bandwidth while containing significant components of the signal and this is performed by the source encoder and source decoder for reversal. As stated in section 2.2, it is assumed that the information source has already been conditioned to bit streams. This implies that source encoding and decoding will be disregarded in this design, thus the information source will be a randomly generated bit stream. 

The channel encoder is concerned with protecting the signal from channel noise, detecting and correcting errors present in the received signal through Forward Error Correction codes. This is achieved by introducing redundancy to the message at the encoder to ensure reliable transmission of information. Redundancy is exploited at the decoder to detect and correct any errors present in the message. 

Modulation involves impressing the codeword on a sinusoidal carrier and generating symbols which make it possible for efficient transmission over a noisy channel. Demodulation performed on the receiver end extracts the codeword from the symbols.

In a typical digital communication system, the symbols are transmitted through the channel after modulation. In this case, it goes through the MIMO architecture before being transmitted as seen in Figure \ref{fig:1}. There are types of MIMO techniques which arrange the symbols from a serial to parallel layout to allow symbols to be transmitted over multiple transmit and receive antennas. This reduces fading effects that the signal might encounter during transmission.

The channel is modelled in such a way that it introduces Rayleigh fading effects as well as Additive White Gaussian Noise.

\section{Design Methodology}
This section specifies and justifies the type of methods for each block that is illustrated in Figure \ref{fig:1}. Detailed mathematical computations for the respective blocks are provided.

\subsection{Forward Error Correction}
As mentioned earlier, wireless channels are prone to certain types of disturbances such as noise and fading. As a result, the received signal will contain errors. To protect the signal against errors, FEC codes are used. FEC codes add parity bits to the message in transit to introduce redundancy. The combination of the message and the parity bits are referred to as the \textit{Codeword}. FEC codes enable error detection and correction on the received codeword. Nevertheless, the codes have limitations to how many errors can be corrected. Error correction codes are classified into two categories, Convolutional Codes and Linear Block Codes. Turbo codes are a type of Convolutional Codes and BCH, RS and LDPC codes are types of Linear Block Codes. All the listed codes are capable of multiple error correction but differ in terms of complexity and implementation.

Turbo codes have been found to perform better than the listed codes because of its ability to provide exceptional coding gains and also perform close to the Shannon Limit \cite{18},\cite{18_7}. Turbo codes usually use large frames of more than 1000 bits since short frames introduce delays at the decoder \cite{chris}. The main disadvantage of using large frames is a highly complex decoder. Similarly, LDPC codes perform well, it is also known that the LDPC codes have a relatively complex decoder \cite{7online}.

A lot of research has been done to compare the performance between RS codes and BCH codes\cite{22}, \cite{9}, \cite{24}. It is found that RS codes are more suitable for concatenated codes and correcting burst errors. BCH codes on the other hand perform better than RS codes in Rayleigh Fading Channels with binary configurations \cite{22}. For this reason BCH codes are chosen for the design of the codec.

\subsubsection{BCH Encoder}
A BCH(\textit{n},\textit{k}) code with \textit{n} = 63, \textit{k} = 45 and a code rate \textit{r} = 0.714 is used since BCH codes with low redundancy and low error correction capability perform better \cite{14_7}. This is due to the fact that longer codewords require more bandwidth and have a high probability of errors. The BCH(63,45) code is capable of correcting a maximum of \textit{t} = 3 errors for a positive integer \textit{m}, where \(m\geq3\) and \(t<2^{m-1}\). The parameters of a BCH(\textit{n},\textit{k}) code are described by equations 1, 2 and 3 as follows:

\begin{equation}
	n = 2^{m}-1
\end{equation}
\begin{equation}
	n-k\leq mt
\end{equation}
\begin{equation}
	d_{min} \geq 2t -1
\end{equation}

wherein \textit{n} is the length of the codeword, \textit{k} is the length of the input message and \(d_{min}\) is the minimum distance. To encode the message \(m(x)\), the BCH encoder employs a generator polynomial \(g(x)\) defined by roots over the Galois Field GF($2^{m}$). Equation 4 describes \(g(x)\) for BCH(64,45).

\begin{equation} 
	g(x)= LCM\{\sigma_{1}(x),\sigma_{2}(x),... \sigma_{2t}(x)\}
\end{equation}

\begin{equation*} 
g(x)  =1 + x + x^2 + x^3 + x^6 + x^7 + x^9 + x^{15} + x^{16} + x^{17} + x^{18}
\end{equation*}

with \(\sigma_{i}(x)\) being the minimal polynomial. The remainder \(r(x)\) as depicted in Equation 5 is then calculated to find the parity check bits.

\begin{equation}
	r(x) = mod(m(x), g(x)) 
\end{equation}

Finally, the codeword \(c(x)\) is found using Equation 6 below:

\begin{equation}
	c(x) = r(x) + m(x)
\end{equation}

The BCH Encoder is the the first object that affects the data rate \textit{$R_{b}$} of $10Mbits/sec$ coming from the source. The resulting data rate after encoding is completed is presented by the equation below:

\begin{equation}
	R_{bfec} = \frac{10Mbits/sec}{r} = 14Mbits/sec
\end{equation}

wherein r is the code rate as described above.

\subsubsection{BCH Decoder}
To retrieve the original message from the received codeword, all the syndromes of the codeword are computed. If any of the syndromes are zero, it means there is no error in the received codeword hence no further action is needed, else the errors need to be located and corrected. This would then be followed by determining the error location polynomial using the syndromes by applying the Berlekamp-Massey Algorithm. Once the error location polynomial has been found, errors are located and corrected.

\subsection{Digital Modulation}
In a digital communication system, the binary or digital data has to be translated to an analogue waveform before transmission occurs. This translation is performed in digital modulation by impressing the message signal on a sinusoidal carrier waveform with a carrier frequency \text{$f_c$}. Modulation gives the message signal immunity to noise in a channel. It moves the spectrum of the baseband signal to frequencies close to that of the carrier frequency which is generally high frequencies in the M-GHz ranges. However, the translated baseband signal still has to adhere to the power constraints of the signal such that there is no interefernce between users within that spectrum \cite{book10}. The effect of noise immunity minimizes the Bit Error Rate (BER) while conserving the power and bandwidth. The resulting modulated signal is a sequence of sinusoids which can be expressed as complex signals also referred to as \textit{Symbols}. Modulation schemes are determined by which property of the carrier has been modified. These properties are the amplitude, frequency and phase of the carrier.

\subsubsection{Amplitude Phase Keying (ASK)}
In ASK Modulation the codeword is mapped to only the amplitude of the carrier by modifying it based on the state of the input codeword. ASK works in such a way that if the input is a 0 then the modulated signal will also be a 0, else if the input is a 1 then the modulated signal will be a sinusoid with amplitude \textit{A}. The advantage of using ASK is that it requires less bandwidth of \(W = 2f_b\) and also has a relatively simple demodulator \cite{61_}. One major flaw with using ASK is that it is highly susceptible to noise which means the transmitted signal will experience high attenuation.  ASK is mostly used in fibre optics communications and this design is focused on wireless systems, therefore it is not applicable. 



\subsubsection{Phase shift Keying (PSK)}
In PSK modulation only the phase of the carrier is changed. If the input is a 1 then the output of the modulator will be in-phase with the carrier, else, the output will be out of phase by $\pi$. PSK is bandwidth efficient because it occupies the same bandwidth as ASK and also offers high noise immunity \cite{61_}. In an M-ary PSK modulator, the unique patterns of bits are mapped to different carriers with different phases. Since PSK has a circular strcuture, all points are the same distance from the origin. Therefore, the higher the modulation order, the closer the constellation points are to each other. This modulation scheme has a tradeoff between increased bit rates (large M) and an increase in the probability of errors in  de-mapping the symbols due to a compact constellation structure \cite{unknownBook}. PSK is also known to have a complex demodulator. The applications of PSK include bluetooth technology, WLAN and radio frequency identification \cite{64}.

\subsubsection{Frequency Shift Keying (FSK)}
FSK only modifies the frequency of the carrier to represent the codeword. If a 1 is transmitted a different carrier frequency \(f_{c1}\) is used and if a 0 is transmitted \(f_{c2}\) is used. Even though FSK has high noise immunity and a moderate complex demodulator compared to PSK, it requires twice the bandwidth (
\(W = 4f_b\)) of PSK and ASK. Considering that bandwidth is a sacred commodity, FSK is not feasible. One major drawback with FSK is that it provides poor spectral efficiency \cite{smith}.

\subsubsection{Quadrature Amplitude Modulation (M-QAM)}
M-QAM modulation is one of the mostly used modulation schemes because of its ability to provide high spectral efficiencies, high transmission rates and energy efficiency \cite{65_10}, \cite{65_11}. For this reason it will be used in the design.

Quadrature Amplitude Modulation is a hybrid of ASK and PSK modulation schemes \cite{unknownBook}. The modulated signal is produced by initially applying a serial to 2-parallel word generator. The parallel sets of data will then be used to modulate the In-phase (I) and Quadrature (Q) components with the same carrier frequency and different amplitudes \cite{65}. Both these components are added to form the modulated signal \textit{s(t)} as illustrated in Figure 2.

\begin{figure}[h!]
	\centering
	\includegraphics[scale=0.53]{QAM.png}
	\caption{QAM Modulator}
	\label{fig:1}
\end{figure}{}

The resulting modulated signal is defined in Equation 8 which can further be expressed as a complex symbol, Equation 9.

\begin{equation}
	s(t) = Acos(2\pi f_{ct}) - Bsin(2\pi f_{c}t)
\end{equation}
\begin{equation}
s = A - jB
\end{equation}

In an M-QAM modulation scheme there are M possible symbols which are used to represent a codeword. Each symbol contains \(k = log_{2}M\) bits. Table 1 shows which M-QAM modulation orders will be investigated as well as the symbol rates and spectral efficiency of the system with the respective modulation scheme. The symbol rate \(R_s\) is determined using Equation 10 below:

\begin{equation}
	R_s = \frac{R_b}{rlog_{2}M}
\end{equation}

wherein r is the code rate of the BCH encoder, \(R_b\) is the information source data rate and M is the modulation order. The spectral efficiency of the system thus far in the absence of MIMO is described in Equation 11. The value of \(\alpha_m\) is assumed to be 1 at this stage implying that there is only 1 transmit and receive antenna.

\begin{equation}
	S_E = \frac{rlog_{2}M\alpha_m}{2} 
\end{equation}

\begin{table}[]
	\centering
	\caption{M-QAM Parameters}
	\label{1}
	\begin{tabular}{|c|l|l|l|}
		\hline
		\textbf{Modulation Order (M)}       & 16                       & 64                        & 128                    \\ \hline
		\textbf{Symbol Rate (Msymbols/sec)} & \multicolumn{1}{r|}{3.5} & \multicolumn{1}{c|}{2.33} & \multicolumn{1}{c|}{2} \\ \hline
		\textbf{Spectral Efficiency ($S_E$)}  & 1.43                     & 2.14                      & 2.5                    \\ \hline
	\end{tabular}
\end{table}

From table 1 it can be seen that 16-QAM at this stage in the design does not meet the success criteria as specified in section 2.1.

\subsubsection{M-QAM Demodulation}
To recover the original codeword, the M-QAM demodulator uses a coherent demodulator. The demodulator makes use of 2 cascaded mixers, lowpass filters and a symbol demapper \cite{B15}. Two carrier  signals are fed into the mixer and its outputs are sent to lowpass filters. The lowpass filter will attenuate the noise introduced during transmission. The symbol demapper will try to map the resulting symbol to one of the possible unique patterns of bits. The decision used to map the symbol is a Hard decision algorithm. In a Hard decision algorithm, the Hamming distance between the received codeword and all the possible patterns of bits are calculated. The pattern which yields the minimum hamming distance is chosen.

\subsection{Rayleigh Fading and AWGN Channel}
It is known that during data transmission the data is affected by the noise and fading effects that are present in the wireless channel. Fading can be described as the rapid changes of the channel state which can be caused by various factors such as the weather \cite{3}. Rapid fluctuations can cause the signal to be attenuated also affecting the phase hence introducing errors. To model the behaviour of these fluctuations, fading coefficients are constructed using two Gaussian random variables \textit{A} and \textit{B} as seen in Equation 12 which both follow a Rayleigh distribution with 0 mean and a variance of 1.

\begin{equation}
	h = A+jB
\end{equation}

The fading effects on the received signal are multiplicative as presented in Equation 13, where \textit{n} is an Additive White Gaussian Noise. Another assumption that is made is that the Channel State Information is already known at both the transmitter and receiver, therefore the fading coefficients will be generated using two different random Gaussian variables.

\begin{equation}
	y = hx +n
\end{equation}

In this design it is assumed that the channel coherence time is 1ms, in other words the channel state will change every 1ms.
\subsection{MIMO}
Multiple Input Multiple Output systems use multiple antennas at the transmitter and at the receiver. MIMO architecture is in such a way that high data rates and increased channel capacity is achieved. This is because of its independent parallel subchannel layout and multipath transmission \cite{50} as indicated in Figure 1. In multipath transmission the same symbol is transmitted in multiple paths thus experiencing different disturbances and fading effects. This gives the system the ability to better retrieve the symbol at the receiver closest to the one that was transmitted. This retrieval would be based on the version of the symbol that experienced minimal disturbances and fading effects. MIMO therefore provides link reliability through multiple antennas and multiple paths. One major challenge with multipath transmission is that symbols arrive at the transmitters at same time causing intersymbol interference (ISI) \cite{50}. ISI is also a consequence of delayed signals from previous transmissions, making it difficult to distinguish the symbols as they will be overlapping. The following equations shows the resulting received signal when ISI occurrs:

\begin{equation}
	Y_Q = \displaystyle\sum_{i=1}^{L} h(i)[A_{i}cos(2\pi f_c \tau_{i}) -B_{i}sin(2\pi f_c \tau_{i})]
\end{equation}
\begin{equation}
Y_I = \displaystyle\sum_{i=1}^{L} h(i)[B_{i}cos(2\pi f_c \tau_{i}) + A_{i}sin(2\pi f_c \tau_{i})]
\end{equation}

wherein L is the number of received signals and h is the randomly generated Rayleigh fading with 0 mean and a variance of 1 as described in section 4.3.

\subsubsection{Space Time Block Codes (STBC)}
STBC is a MIMO scheme that transmits copies of the same symbol through multiple transmit antennas \cite{alsahlanyperformance}. Different versions of the the same symbol are received to get a more reliable version of the transmitted signal. It was found that with STBC, the probability of loosing the transmitted signal decreases exponentially \cite{424}. STBC scheme is used to increase the diversity gain of the system and thus does not provide any improvement towards the spectral efficiency. This design is aimed at achieving high spectral efficiencies  therefore the STBC scheme is not chosen.

\subsubsection{Spatial Multiplexing}

Spatial Multiplexing is different from STBC codes such that instead of sending multiple copies of the same symbol, several different symbols are transmitted at the same time through independent transmit antennas and channels \cite{slides}. This technique allows an increase in data throughput by increasing channel capacity at the same  bandwidth and also an increase in spectral efficiency. For this reason, this MIMO scheme will be adopted in this design.

The following equations shows the relation between the received signal vector for a MIMO architecture with \(N_t\) transmit antennas, \(N_r\) receive antennas and a matrix of fading coefficients H of size \(N_r\times N_t\). \(\vec{n}\) is a vector of AWGN noise variables. 

\begin{equation}
\vec{y} = H\vec{x}+\vec{n}
\end{equation}

\begin{equation}
\quad
\begin{bmatrix}
y_1  \\
y_2  \\
\vdots \\
y_{N_r}
\end{bmatrix}
\quad =	\quad
	\begin{bmatrix}
	h_{11} & h_{12} & \ldots &h_{1N_t}\\
	h_{21} & h_{21} & \ldots &h_{2N_t} \\
	\vdots  & \vdots & \ddots& \vdots \\
	h_{N_r 1} &h_{N_r 2} & \ldots&h_{N_r N_t}
	\end{bmatrix}
	\quad \\
	\quad
	\begin{bmatrix}
	x_1  \\
	x_2  \\
	\vdots \\
	x_{N_t}
	\end{bmatrix} + \quad
	\begin{bmatrix}
	n_1  \\
	n_2  \\
	\vdots \\
	n_{N_r}
	\end{bmatrix}
\end{equation}

The spectral efficiency of the system will now be affected by how many transmit antennas are employed. Equation 11 will be used to calculate the spectral efficiency , however the value of \(\alpha_m\) will now be equal \(N_t\). The performance of a  \(2\times2\) MIMO system will be investigated in this design. Table 2 shows the overall spectral efficiency of the system for the different modulation schemes with MIMO. It can be seen that all the modulation schemes meet the specified success criteria.

\begin{table}[h!]
	\centering
	\caption{Spectral Efficiency of the Investigated M-QAM schemes}
	\label{1}
	\begin{tabular}{|c|l|l|l|}
		\hline
		\textbf{Modulation Order (M)}      & 16   & 64   & 128 \\ \hline
		\textbf{Spectral Efficiency ($S_E$)} & 2.85 & 4.28 & 5   \\ \hline
	\end{tabular}
\end{table}

\subsection{MIMO Receiver and Equalization}
In MIMO systems all the receive antennas will receive signals from all the transmit antennas, therefore the receieved signal at each antenna will be a combination of all the transmitted symbols \cite{BOOK8}. To detect the desired transmitted signal, a method called MIMO equalization is employed. This method tries to minimize other interfering symbols. Since the receiver is linear a linear equalizer must be used. There are two types of linear equalizers, one being the Zero Forcing (ZF) and the other Minimum Mean Square Error (MMSE). ZF is not efficient because it amplifies the noise if the value of any of the fading coefficients is too small \cite{46}. MMSE on the other hand minimizes ISI to an extent as well as the noise power \cite{41}. This is achieved by minimizing the mean square error with an equalization matrix \textbf{W} in Equation 18. MMSE will thus be adopted in this design. 

\begin{equation}
	W = {[H^{H}H + N_o I]}^{-1} H^H
\end{equation}

The estimated received signal is expressed in the equation below:

\begin{equation}
	\tilde{x} = W\vec{y}
\end{equation}
\section{Simulation Results}
\section{Critical Analysis}

\section{Future Recommendations}
\section{Impacts}

\section{Conclusion}
\newpage
\bibliographystyle{witseie}
\bibliography{references}


\newpage 
\begin{appendices}
	
	\section{Impact Analysis of Proposed Wireless MIMO Codec}
	
	A MIMO codec is a wireless communication system that takes in data, compresses it, adds protection against any disturbances for reliable transmission by using multiple transmit and receive antennas. All the processes mentioned are then reversed to retrieve the data that was transmitted. This enables communication systems such as cellphones to carry and process more information seamlessly at the same time with high transmission rates. The data that is transmitted using MIMO technology will go through multiple paths to improve the communication link reliability. One advantage with MIMO systems is that it is able to choose from the multiple received signals the one that most represents a reliable version of the transmitted information. This further improves the quality and strength of the information as well as communication.
	
	One of the major challenges experienced by Network Providers in South Africa is that the spectrum that they are using for cellular communication is currently being used close to full capacity. This creates problems within the communication links which causes calls to be dropped and delayed transmission of information. To overcome this challenge, MIMO technology can be used because it will allow more people to communicate at the same time hence increasing the capacity of the communication channel whilst allowing data to be transmitted reliably at very high rates.
	
	\subsection{Social Impacts}
	Wireless technology has formed a great part of our lives making it easy to for people to communicate across different parts of the world. It has made it possible for people to interact through social media platforms, which can grow an individuals social network and help people aware of what is happening around them. Multiple people can now communicate using group video calls at improved quality. Nowadays people do not have to be physically present at interviews, it is easily achieved through a simple video call which saves people money and time. During natural disasters, mobile stations are used to provide communication between the people who are affected to make it possible for them to receive help as quick as possible. 
	
	Information is now easily accessible for people to educate themselves. Entertainment is also now available to our fingertips, people can play online games, stream videos or movies at very high download speeds and great quality. One of the advantages to data protection is that hackers will have a difficult time setting up their receivers since information is broken up and sent through multiple antennas.
	
	\subsection{Economic Impacts}
	A lot of people are now able to start and advertise their business through social media platforms which is a great way to boost the economy by reducing unemployment rates. Businesses are now able to safe a lot of money on travelling by using conference calls to get hold of their customers or business partners. The use of MIMO by service providers can help reduce the high data costs since the communication systems will now have have a high spectral efficiency. However to upgrade base stations to use this type of technology will be very costly.
	
	MIMO will make it possible to have a reduced number of base stations, thus saving companies a lot of money \cite{50}.
	\subsection{Environmental Impacts}
	As stated in the economic impacts that it will be possible to have a reduced number of based stations, this will allow people to have more land to build on for various activities. Since upgrades will need to be done to base stations and other types of infrastructures, this means most hardware will no longer be used which adds to waste that is non-biodegradable. To reduce this addition of such waste termed e-waste it is advisable to use software defined systems that will only need updates and not new hardware upgrades.
	
	In conclusion, MIMO technology will make a great impact to peoples lives, the economy and the environment.
	
\end{appendices}


\end{document}
